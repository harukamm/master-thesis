\chapter {結論}\label{chap:conclusion}

ビジュアルプログラミング環境構築のためのライブラリBlocklyをベースにして,
OCamlプログラムを組み立てるビジュアルプログラミング環境,OCaml Blocklyを開発した.
束縛変数,let多相を含む型システムを実装し,シンタックスエラー,型エラー,Unbound valueエラーを起こすプログラムの組み立てを拒否するユーザインタフェースを構築した.
変数ブロックを用いた自由なブロックの組み立てを行うことができる,スコープチェックの付いた空間,スコープお砂場を実装した.
ユーザが理解しやすいエラー出力や,
OCaml言語に慣れるために必要な基礎的な構文のサポート,
テキストベースのプログラミングへの導入となるブロックとテキストとの相互変換を行った.

実使用に向けた課題は第\ref{sec:seiri}節にてまとめたが,それとは独立して,OCaml Blocklyにはさらに高度な拡張に向けた展望がある.
例えば,Try OCamlを用いたブラウザ上でのプログラムの実行である.
他にも,型デバッガ\cite{10.1007/978-3-642-41582-1_12}やステッパ\cite{Stepper}といった本学の授業で既に用いられているツールをOCaml Blocklyに導入することが考えられる.
型の整合性を壊すようなコネクタを接続させようとしたときに,ブラウザ上に型デバッガを起動すれば良い.
そうすれば,本論文が第\ref{fun:message}節で行なった,2つの型がなぜ合わないかのエラー出力だけでなく,ユーザの意図したプログラムの型が破綻した原因は
どの箇所にあったのか,という型の間違いを発見することができる.
また,ステッパと対話することで,ブロック上でのステップ実行が実現できるだろう.
ステッパの出力したステップ実行後のプログラムに合わせてブロックを更新し,実行中の箇所をハイライトすれば,初学者の理解を更に深められると考えている.
 

これからOCaml Blocklyを実使用していくことに向けて,本システムをさらに拡張,改善していきたい.
OCaml Blocklyの実装は,https://github.com/harukamm/ocaml-blocklyにて公開されている.
