\chapter{序論}\label{chap:intro}

関数型言語の初学者が関数型言語を学び始めたとき,
数々の本質的でない問題に陥りやすい.
変数に何らかの値を再代入しようとしてシンタックスエラーを起こしたり,
Int型とFloat型を持つ値を足し合わせようとして型エラーになってしまうことがある.
関数型言語のシンタックスや静的型付けに慣れていない初学者にとって,
テキストエディタは入力の自由度が高すぎるため,
エラーの原因を特定するのはしばしば困難である.

プログラミング初学者がプログラミング学習のために用いるツールの1つにビジュアルプログラミング環境がある\cite{Viscuit,Scratch,Snap}.
ビジュアルプログラミング環境では,テキストではなく,視覚的なオブジェクトを組み合わせることでプログラミングを行う.
例えば,近松ら\cite{HaskellVP}は,
変数や構文をノードで表し,ノードを線で繋いで値の入出力の関係を制御することによって,
Haskellプログラムをグラフとして表現する.

一方で,Googleの提供するBlockly \cite{Blockly}はブラウザにおけるビジュアルプログラミング環境構築のためのライブラリとして配布されているオープンソースプロジェクトであり,完全にクライアントサイドのみで動く.
ブロックの描画や移動,組み立てなどといったユーザインタフェースに関わる実装はライブラリ側が全て行なっているため,
ブロックに表示する文字や入出力の関係を定義するだけで,カスタマイズされたブロックを持つビジュアルプログラミング環境を作ることができる.
実装言語はJavaScriptであり,ブロックの描画にはSVGやCSSが活用されている.
また,Blocklyの標準のブロックであれば,ユーザが組み立てたブロック群から,JavaScript,Pythonなどといったスクリプト言語に変換することができる.

% グラフ型のビジュアルプログラミング言語との比較.なぜブロック型を採用したか.見た目的に取っつきやすい.Blocklyは開発者向けなので改造しやすい.
% 構文木とブロックが一対一対応であるのが視覚的にわかりやすく,シンタックスエラーが起こり得ない.
% ブロック型は関数型言語の上で重要な入出力の関係を載せるのに向いていない\cite{HaskellVP}が,(なにか主張を書く)
% なぜOCamlなのか
% sytaxになれる
本研究では,既に洗練されたユーザインタフェースが用意されていることや,構文木と一対一対応をしていてテキストベースのコードを連想しやすいことから,
BlocklyをベースとしてOCamlビジュアルプログラミング環境を実装し,これをOCaml Blocklyと名付けた.
OCaml Blockly では,不正なプログラムを表すブロックを組み立てることをユーザインタフェースによって制限する.
また,各ブロックにその構文が持つキーワードを印字し,ブロックが持つ型を形や色で表して視覚的に出力する.

本システムが想定するユーザの一例としては,本学のOCamlを用いた授業の履修者が挙げられる.
履修者は毎年の傾向として,命令型言語であるC言語を用いたプログラミングを別の授業で経験したことはあるものの,関数型言語の経験がない人がほとんどである.
履修者は授業の中で,短期間のうちにOCamlプログラミングに慣れる必要がある.
授業はシンタックスハイライト機能の付いたテキストエディタを使って行われるが,C言語とOCamlの構文が大きく異なるために,正しい構文を書けずにシンタックスエラーに詰まってしまう人が多い.
また,強い静的型付けに慣れていないために,型を捉えることが難しい傾向がある.
そのようなユーザがOCaml Blocklyの元でOCamlプログラミングを行うことで,以下の利点を得られることを期待している.
\begin{itemize}
  % IDEのはなし
  \item OCamlの構文を覚えていなくても,シンタックスエラーに陥ることなくプログラムが組める.
  \item 不正なプログラムを組むことができないため,複数のエラーが同時に起きることがなく,テキストベースよりもストレスが少なく,OCamlプログラミングの本質を優先的に学べる.
  \item 1つ1つの式が持つ型が見えることで,型付けの関係を理解しやすい.
\end{itemize}
%関数型言語初学者がOCamlを学習する際の導入段階に使用し,シームレスにテキストベースによるプログラミングに繋がるようにすることが狙いである.

また,ブロックによるプログラミングによってOCamlの言語仕様に慣れ親しんだのち,シームレスにテキストベースによるプログラミングに繋がるように,本研究ではブロックとOCamlコードの相互変換の実装を行った.

本論文の構成を以下に示す.まず,{\bf 第\ref{chap:blocklyWord}章}でBlockly固有の用語を共有する.
{\bf 第\ref{chap:senko}章}にて関連研究とその問題点について分析し,その問題点を踏まえて本研究が将来的に達成すべき目標と,
本論文で達成する目標を示す.
次に,{\bf 第\ref{chap:features}章}で本論文が実装した主な機能を紹介し,
その各機能の実装の詳細を{\bf 第\ref{chap:impl}章}にて説明する.
{\bf 第\ref{chap:converter}章}でブロックとOCamlコードとの相互変換を行うためのシステムの構成を述べる.
{\bf 第\ref{chap:todo}章}で本システムの現状をまとめ,最終的な目標の達成のために実現されるべき課題を整理し,考察する.
最後に,{\bf 第\ref{chap:conclusion}章}にて本論文をまとめ,今後の展望について言及する.
